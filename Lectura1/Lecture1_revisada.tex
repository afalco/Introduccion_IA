\documentclass[10pt]{beamer}

\usepackage[utf8]{inputenc}
\usepackage[T1]{fontenc}
\usepackage[spanish]{babel}
\usepackage{lmodern}
\usepackage{amsmath, amssymb}
\usepackage{graphicx}
\usepackage{tikz}
\usetikzlibrary{positioning, arrows.meta}
\usepackage{hyperref}
\usepackage{xcolor}

\title{Introducci\'on al Procesado de Imagen Digital}
\author{Antonio Falc\'o y Juan Pardo}
\institute{Introducci\'on a la Inteligencia Artificial}
\date{Primera lectura del curso}

\begin{document}

% =====================
% Portada
% =====================
\begin{frame}
  \titlepage
\end{frame}

% =====================
% Nueva slide introductoria
% =====================
\begin{frame}{De p\'ixel a modelo de IA}
\centering
\begin{tikzpicture}[node distance=1.2cm and 1.0cm, every node/.style={font=\scriptsize}]
  % Imagen
  \node[draw, fill=blue!10, minimum width=2cm, minimum height=1cm] (img) { \tiny\textbf{Imagen digital}};
  % Matriz
  \node[draw, fill=green!10, minimum width=2cm, minimum height=1cm, right=of img] (mat) { \tiny\textbf{Matriz de p\'ixeles}};
  % Datos
  \node[draw, fill=yellow!10, minimum width=2cm, minimum height=1cm, right=of mat] (data) { \tiny\textbf{Datos num\'ericos}};
  % Modelo
  \node[draw, fill=red!10, minimum width=2cm, minimum height=1cm, right=of data] (model) { \tiny\textbf{Modelo de IA}};

  % Flechas
  \draw[-{Latex[length=3mm]}] (img) -- (mat);
  \draw[-{Latex[length=3mm]}] (mat) -- (data);
  \draw[-{Latex[length=3mm]}] (data) -- (model);

  % Iconos simb\'olicos
  \node[below=0.2cm of img] { \tiny C\'amara / Sensor};
  \node[below=0.2cm of mat] { \tiny Estructura bidimensional};
  \node[below=0.2cm of data] { \tiny Normalizaci\'on y vectorizaci\'on};
  \node[below=0.2cm of model] { \tiny Clasificaci\'on / Segmentaci\'on};
\end{tikzpicture}
\vspace{4mm}
\begin{block}{Idea clave}
Una imagen pasa de ser una representaci\'on visual a un conjunto de datos que puede aprender un modelo de inteligencia artificial.
\end{block}
\end{frame}

% =====================
% 1. Aprendizaje e IA
% =====================
\begin{frame}{El aprendizaje en inteligencia artificial}
\begin{itemize}
  \item El aprendizaje es el proceso mediante el cual un sistema mejora su desempe\~no a partir de la experiencia (datos).
  \item En IA, se entrena un modelo matem\'atico alimentado con datos para realizar tareas: reconocer patrones, tomar decisiones o predecir resultados.
  \item En el contexto del dise\~no de producto, el aprendizaje permite analizar im\'agenes, optimizar procesos o generar dise\~nos autom\'aticamente.
\end{itemize}
\vspace{3mm}
\begin{block}{Esquema general}
\centering
Datos $\rightarrow$ Algoritmo de entrenamiento $\rightarrow$ Modelo entrenado $\rightarrow$ Predicciones
\end{block}
\end{frame}

% =====================
% 2. Procesado de imagen digital
% =====================
\begin{frame}{\'Que es una imagen digital?}
\begin{itemize}
  \item Una imagen digital es una representaci\'on num\'erica de la intensidad de luz capturada por un sensor o una c\'amara.
  \item Cada elemento de la imagen se llama \textbf{p\'ixel} y almacena informaci\'on sobre el color o la intensidad.
  \item Matem\'aticamente, una imagen se representa como una funci\'on
  $$f: \{0,\ldots,W-1\}\times\{0,\ldots,H-1\}\to\mathbb{R}$$
  donde $f(x,y)$ es la intensidad en la posici\'on $(x,y)$.
\end{itemize}
\end{frame}

% =====================
% Figura: C\'amara oscura
% =====================
\begin{frame}{Modelo de c\'amara oscura}
\centering
\begin{tikzpicture}[scale=0.9, every node/.style={font=\small}]
  \draw[thick] (0,0) rectangle (5,3);
  \filldraw[fill=gray!10] (0,0) rectangle (5,3);
  \draw[fill=white] (0.2,1.5) circle (0.1);
  \node[left] at (0,1.5) {Orificio};
  \draw[-{Latex[length=3mm]}] (0.2,1.5) -- (2.5,1.8);
  \draw[-{Latex[length=3mm]}] (0.2,1.5) -- (2.5,1.2);
  \node at (4,1.5) {Pantalla};
  \draw[dashed] (2.5,0) -- (2.5,3);
\end{tikzpicture}
\vspace{2mm}
\begin{block}{Idea clave}
La imagen es una proyecci\'on del mundo real en una superficie sensible a la luz.
\end{block}
\end{frame}

% =====================
% Representaci\'on num\'erica
% =====================
\begin{frame}{Representaci\'on num\'erica de una imagen}
\begin{itemize}
  \item Cada p\'ixel se asocia a un valor entero o real (seg\'un la profundidad de bits).
  \item Im\'agenes en escala de grises: una sola matriz $f(x,y)$. \\
  Im\'agenes en color: tres matrices (canales R, G, B).
\end{itemize}
\begin{block}{Ejemplo}
Formatos comunes: RAW, BMP, JPEG, PNG.
\end{block}
\end{frame}

% =====================
% Figura: Canales RGB
% =====================
\begin{frame}{Canales de color RGB}
\centering
\begin{tikzpicture}[scale=0.9]
  \fill[red!60] (0,0) rectangle (1,1);
  \fill[green!60] (1.2,0) rectangle (2.2,1);
  \fill[blue!60] (2.4,0) rectangle (3.4,1);
  \node[below] at (0.5,0) {R};
  \node[below] at (1.7,0) {G};
  \node[below] at (2.9,0) {B};
  \draw[thick, ->] (3.6,0.5) -- (4.6,0.5) node[right,xshift=15mm]{Combinaci\'on RGB};
  \fill[red!60!green!60!blue!60] (5,0) rectangle (6,1);
\end{tikzpicture}
\vspace{2mm}
\begin{block}{Observaci\'on}
Cada canal almacena una componente del color; la combinaci\'on reconstruye la imagen original.
\end{block}
\end{frame}

% =====================
% 3. Ejemplos Python
% =====================
\begin{frame}[fragile]{Primeros pasos en Python}
\begin{verbatim}
import matplotlib.pyplot as plt
import numpy as np

img = plt.imread('imagen.jpg')
print('Dimensiones:', img.shape)
print('Tipo:', img.dtype)

plt.imshow(img)
plt.title('Imagen original')
plt.axis('off')
plt.show()
\end{verbatim}
\vspace{2mm}
\begin{block}{Salida esperada}
Una imagen cargada y mostrada en pantalla, junto con su tama\~no y tipo.
\end{block}
\end{frame}

% =====================
% Ejercicio guiado
% =====================
\begin{frame}[fragile]{Ejercicio guiado: transformaciones}
\begin{verbatim}
# Extraer una regi\'on (slicing)
plt.imshow(img[100:400, 200:600])
plt.title('Recorte de la imagen')
plt.show()

# Invertir verticalmente
plt.imshow(img[::-1, :, :])
plt.title('Imagen invertida verticalmente')
plt.show()
\end{verbatim}
\begin{block}{Para pensar}
\begin{itemize}
  \item \textbf{¿Qu\'e ocurre si inviertes horizontalmente?}
  \item \textbf{¿C\'omo podr\'ias mostrar solo el canal rojo?}
\end{itemize}
\end{block}
\end{frame}

% =====================
% Cierre
% =====================
\begin{frame}{Conclusi\'on}
\begin{itemize}
  \item Las im\'agenes digitales son datos num\'ericos estructurados como matrices.
  \item Comprender su representaci\'on es esencial antes de aplicar t\'ecnicas de IA o aprendizaje profundo.
  \item En la siguiente sesi\'on: filtros, convoluciones y transformaciones b\'asicas.
\end{itemize}
\vfill
\centering\textit{\small Antonio Falc\'o y Juan Pardo -- CEU UCH}
\end{frame}

\end{document}
