% !TEX program = pdflatex
\documentclass[10pt,aspectratio=169]{beamer}

% ====== Idioma y codificación ======
\usepackage[spanish]{babel}
\usepackage[utf8]{inputenc}
\usepackage[T1]{fontenc}
\usepackage{lmodern}

% ====== Tema moderno ======
\usetheme{metropolis}
\definecolor{ceuBlue}{RGB}{0,56,168}
\definecolor{ceuGreen}{RGB}{87,171,39}
\setbeamercolor{structure}{fg=ceuBlue}
\setbeamercolor{progress bar}{fg=ceuGreen,bg=ceuBlue!20}
\setbeamercolor{frametitle}{bg=ceuBlue!6}

% ====== Paquetes ======
\usepackage{amsmath,amssymb}
\usepackage{graphicx}
\usepackage{booktabs}
\usepackage{hyperref}
\usepackage{tikz}
\usetikzlibrary{arrows.meta,positioning,calc}

% ====== Datos de la presentación ======
\title{Introducción a la Inteligencia Artificial\newline\large (Visión y aprendizaje automático desde cero)}
\author{Antonio Falcó \& Juan Pardo}
\institute{Máster en IA y Fabricación Aditiva — ESET}
\date{\today}

\begin{document}

% ========== Portada ==========
\begin{frame}[plain]
  \titlepage
\end{frame}

\begin{frame}{Mapa del curso}
  \tableofcontents
\end{frame}

% =============================================================
\section{Objetivos y estructura}
% =============================================================

\begin{frame}{Objetivo del curso}
  \begin{block}{Propósito}
    Aprender los principios básicos de la inteligencia artificial a través de ejemplos visuales y prácticos, sin necesidad de conocimientos previos avanzados de matemáticas o programación.
  \end{block}
  \vspace{0.4em}
  \textbf{Enfoque}: comprensión intuitiva, uso de herramientas accesibles (Python notebooks, Colab) y aprendizaje progresivo.
\end{frame}

\begin{frame}{Estructura general}
  \begin{itemize}
    \item Semana 1–2: ¿Qué es la IA? — ideas, historia y ejemplos.
    \item Semana 3: Cómo ven las máquinas — imágenes y píxeles.
    \item Semana 4: Aprender a clasificar — introducción a Machine Learning.
    \item Semana 5–6: Redes neuronales con ejemplos sencillos.
    \item Semana 7–8: Proyecto final guiado (clasificar imágenes o detectar objetos).
  \end{itemize}
\end{frame}

% =============================================================
\section{Qué es la Inteligencia Artificial}
% =============================================================

\begin{frame}{¿Qué entendemos por IA?}
  \begin{itemize}
    \item Sistemas que aprenden de los datos y toman decisiones.
    \item Desde los primeros programas de ajedrez hasta los asistentes virtuales actuales.
    \item La IA combina matemáticas, lógica y experiencia.
  \end{itemize}
\end{frame}

\begin{frame}{Ejemplos cotidianos de IA}
  \begin{itemize}
    \item Recomendaciones en redes sociales o plataformas de vídeo.
    \item Asistentes de voz como Alexa o Siri.
    \item Detección de defectos en productos fabricados.
  \end{itemize}
\end{frame}

% =============================================================
\section{Procesamiento de imágenes}
% =============================================================

\begin{frame}{Cómo ven las máquinas}
  \begin{block}{Idea básica}
    Una imagen es una cuadrícula de puntos (píxeles) con valores que representan el color o la luz.\newline
    \textit{Ejemplo: una foto en blanco y negro se convierte en una tabla de números.}
  \end{block}
\end{frame}

\begin{frame}{Pequeños experimentos}
  \begin{itemize}
    \item Cargar y mostrar una imagen en Python (con Colab o Jupyter).
    \item Cambiar brillo y contraste.
    \item Convertir una imagen en blanco y negro.
  \end{itemize}
\end{frame}

% =============================================================
\section{Primeros pasos en Machine Learning}
% =============================================================

\begin{frame}{¿Qué significa aprender de los datos?}
  \begin{block}{Ejemplo intuitivo}
    Si un programa ve muchas imágenes de gatos y perros, puede aprender a diferenciarlos observando los patrones comunes.
  \end{block}
  \begin{itemize}
    \item No se trata de memorizar, sino de generalizar.
    \item Usaremos ejemplos visuales para entender cada concepto.
  \end{itemize}
\end{frame}

\begin{frame}{Tipos de aprendizaje}
  \begin{itemize}
    \item \textbf{Supervisado}: el programa aprende a partir de ejemplos con etiqueta (por ejemplo, “esto es un gato”).
    \item \textbf{No supervisado}: busca patrones o grupos por sí solo.
    \item \textbf{Por refuerzo}: aprende por prueba y error.
  \end{itemize}
\end{frame}

% =============================================================
\section{Redes neuronales desde cero}
% =============================================================

\begin{frame}{La idea de la neurona}
  \begin{block}{Concepto visual}
    Una neurona recibe información, hace un cálculo sencillo y decide una salida.\newline
    En la práctica: suma de valores con diferentes pesos y una regla de decisión.
  \end{block}
\end{frame}

\begin{frame}{Perceptrón — el modelo más simple}
  \centering
  \begin{tikzpicture}[node distance=2cm,>=latex]
    \node[circle,draw,fill=ceuGreen!30,minimum size=10mm] (n) at (0,0) {$f$};
    \draw[-Latex] (-2,0.8)--(n) node[midway,above]{\small $x_1$};
    \draw[-Latex] (-2,0)--(n) node[midway,above]{\small $x_2$};
    \draw[-Latex] (-2,-0.8)--(n) node[midway,below]{\small $x_3$};
    \draw[-Latex] (n)--(2,0) node[midway,above]{\small salida};
  \end{tikzpicture}
  \vspace{0.5em}
  \begin{block}{Función}
    $f(x) = w_1x_1 + w_2x_2 + w_3x_3 + b$  →  si el resultado es positivo, activa la salida.
  \end{block}
\end{frame}

\begin{frame}{Visualizando una red sencilla}
  \begin{itemize}
    \item Varias neuronas conectadas forman una red.
    \item Cada capa transforma los datos un poco más.
    \item Las redes profundas son combinaciones de muchas capas.
  \end{itemize}
\end{frame}

% =============================================================
\section{Proyecto final}
% =============================================================

\begin{frame}{Proyecto: Clasificación de imágenes}
  \begin{block}{Objetivo}
    Entrenar un modelo sencillo que reconozca imágenes (por ejemplo, distinguir flores o tipos de piezas 3D).
  \end{block}
  \begin{itemize}
    \item El proyecto se realiza paso a paso con guía.
    \item Se proporcionan datos y ejemplos de código.
  \end{itemize}
\end{frame}

\begin{frame}{Evaluación}
  \begin{itemize}
    \item Comprensión y claridad en la explicación (40\%).
    \item Corrección del proceso y resultados (40\%).
    \item Participación y trabajo en equipo (20\%).
  \end{itemize}
  \vspace{0.5em}
  \textbf{Importante:} no se evalúa la complejidad del código, sino la comprensión del proceso.
\end{frame}

% =============================================================
\section{Cierre}
% =============================================================

\begin{frame}{Resumen}
  \begin{itemize}
    \item Hemos visto qué es la IA y cómo las máquinas aprenden de ejemplos.
    \item Hemos aprendido a representar imágenes como datos.
    \item Conocemos el funcionamiento básico de una red neuronal.
    \item Próximo paso: aplicar todo en un pequeño proyecto.
  \end{itemize}
\end{frame}

\begin{frame}{Mensaje final}
  \centering
  \Large\textbf{La IA se aprende practicando y explorando.}\\[6pt]
  \normalsize\textit{Comienza con curiosidad, sin miedo a equivocarte.}
\end{frame}

\end{document}